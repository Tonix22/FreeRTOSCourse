\begin{enumerate}[label=\textbf{Week \arabic*:}]
    \item \textbf{Introduction to Real-Time Systems}
    \begin{itemize}
        \item What is an operating system?
        \item Main features of an OS
        \item Safety
        \item Security
        \item What is real time?
        \item Introduction to Real-Time Operating Systems (RTOS) and their relevance in embedded solutions
        \item MIT License, GNU, MIT, and GPL
        \item Understanding Copyleft
        \item Comparison of Open Source Licenses
    \end{itemize}

    \item \textbf{Critical Systems and Applications of RTOS}
    \begin{itemize}
        \item Discussion of RTOS applications in critical systems across various industries:
        \begin{itemize}
            \item Aerospace and Aviation (e.g., General Electric, Hydra, DJI)
            \item Automotive (e.g., Continental, NXP, Avnet)
            \item Healthcare (e.g., Plexus, Baxter)
            \item Power Generation (e.g., Baker Hughes, CERN)
            \item Telecommunications (e.g., Cinvestav, Qualcomm)
            \item Defense and Military (e.g., Hydra, A2E)
        \end{itemize}
    \end{itemize}

    \item \textbf{Continuous Integration and Professional Workflow}     
    \begin{itemize}
        \item Overview of standard industry practices.
        \item Importance of documentation and diagrams in maintaining scalable and maintainable code.
        \item Software Development Guide (SDG)
        \item Key components of an effective SDG.
        \item Software Detailed Document (SDD)
        \item Requirements
        \item Exploring the role and structure of SDD.
        \item Unified Modeling Language (UML)
        \item Development process, good practices, and coding standards
        \item Version control process GIT
        \item Test plan document
        \item Whitebox testing and manual testing
        \item Understanding Continuous Integration
        \item Benefits of integrating CI into the software development lifecycle.
    \end{itemize}

    \item \textbf{Continuous Integration and Professional Workflow (continued)}
    \begin{itemize}
        \item Practical perspective on applying CI strategies in development projects.
    \end{itemize}

    \item \textbf{FreeRTOS Architecture (Part 1)}
    \begin{itemize}
        \item Introduction to heap/stack memory management in RTOS.
        \item Detailed discussion on function pointers and the concept of callbacks.
        \item Overview of interrupt service routines and timer ISR.
    \end{itemize}

    \item \textbf{FreeRTOS Architecture (Part 2)}
    \begin{itemize}
        \item In-depth look at concurrent processing in FreeRTOS:
        \begin{itemize}
            \item Scheduler dynamics
            \item Task management and priority settings
            \item Queues, semaphores, and mutexes
        \end{itemize}
    \end{itemize}

    \item \textbf{Hands-on with ESP32 (Setup and Basic Examples)}
    \begin{itemize}
        \item Introduction to Espressif microcontrollers, with a focus on ESP32.
        \item Setting up the development environment, including Visual Studio Code.
        \item Configuring WSL for the Espressif FreeRTOS SDK.
        \item Practical examples: Running basic GPIO LED control tasks.
    \end{itemize}

    \item \textbf{Advanced Hands-on with ESP32}
    \begin{itemize}
        \item Techniques for effective serial debugging.
        \item WIFI drivers and code examples
    \end{itemize}

    \item \textbf{FreeRTOS Programming (Part 1)}
    \begin{itemize}
        \item Techniques for task creation and management.
        \item Control functions such as delay, suspend, and resume.
        \item FreeRTOS configuration overview
    \end{itemize}

    \item \textbf{FreeRTOS Programming (Part 2)}
    \begin{itemize}
        \item Comprehensive management of queues and semaphores.
    \end{itemize}

    \item \textbf{Midterm Review and Examination}
    \begin{itemize}
        \item Review of all topics covered thus far.
        \item Midterm examination to assess knowledge and practical application.
    \end{itemize}

    \item \textbf{IoT MQTT Programming Part 1 (TCP and WIFI)}
    \begin{itemize}
        \item Introduction to MQTT protocols
        \item Setting up MQTT broker and clients
        \item Implementing MQTT with TCP and WiFi on ESP32
    \end{itemize}

    \item \textbf{IoT MQTT Programming Part 2 (Adafruit IO)}
    \begin{itemize}
        \item Integrating IoT devices with Adafruit IO using MQTT
        \item Developing interactive IoT applications
        \item Practical examples and project ideas
    \end{itemize}

    \item \textbf{Project Design, Requirements, UML and Planning}
    \begin{itemize}
        \item Detailed discussion on final project design and requirements.
        \item Initial project planning and milestone setting.
        \item Utilizing Unified Modeling Language (UML) to plan and model projects.
        \item Creating sequence diagrams and behavioral architectures for real-time systems.
    \end{itemize}

    \item \textbf{FreeRTOS Programming (Part 3)}
    \begin{itemize}
        \item Detailed handling of software timers and event groups.
        \item Advanced scheduling techniques.
    \end{itemize}

    \item \textbf{Final Project workshop (Part 1)}
    \begin{itemize}
        \item Practical implementation of the final projects.
        \item Final presentations and review sessions.
    \end{itemize}

    \item \textbf{Final Project workshop (Part 2)}
    \begin{itemize}
        \item Continued implementation and final adjustments.
        \item Concluding presentations and evaluations.
    \end{itemize}

    \item \textbf{Final Project workshop (Part 3)}
    \begin{itemize}
        \item Completion of project implementations.
        \item Final evaluations and feedback.
    \end{itemize}

\end{enumerate}