\documentclass[10pt]{beamer}
\usepackage[utf8]{inputenc}
\usepackage{graphicx}
\usepackage {mathtools}
\usepackage{hyperref}
\usepackage{hyperref}
\usepackage{utopia} %font utopia imported
\usetheme{CambridgeUS}
\usecolortheme{dolphin}
\hypersetup{colorlinks=true,linkcolor=blue, linktocpage}
% set colors
\definecolor{myNewColorA}{RGB}{118,193,188}
\definecolor{myNewColorB}{RGB}{106,172,150}
\definecolor{myNewColorC}{RGB}{94,150,218}
\setbeamercolor*{palette primary}{bg=myNewColorC}
\setbeamercolor*{palette secondary}{bg=myNewColorB, fg=white}
\setbeamercolor*{palette tertiary}{bg=myNewColorA, fg=white}
\setbeamercolor*{titlelike}{fg=myNewColorA}
\setbeamercolor*{title}{bg=myNewColorA, fg=white}
\setbeamercolor*{item}{fg=myNewColorA}
\setbeamercolor*{caption name}{fg=myNewColorA}
\usefonttheme{professionalfonts}
\usepackage{natbib}
\usepackage{hyperref}
%------------------------------------------------------------
\titlegraphic{\includegraphics[height=1.5cm]{../CommonFigures/Universidad_Panamericana-logo.jpg}}

\setbeamerfont{title}{size=\large}
\setbeamerfont{subtitle}{size=\small}
\setbeamerfont{author}{size=\small}
\setbeamerfont{date}{size=\small}
\setbeamerfont{institute}{size=\small}
\title[Universidad Panamericana]{}
\subtitle{Critical Systems and Applications of RTOS}
\author[]{Name}

\institute[ltonix@up.edu.mx]{Universidad Panamericana}
\date[Presentation \today]
{Presentation \today}

\AtBeginSection[]{
  \begin{frame}
  \vfill
  \centering
  \begin{beamercolorbox}[sep=8pt,center,shadow=true,rounded=true]{title}
    \usebeamerfont{title}\insertsectionhead\par%
  \end{beamercolorbox}
  \vfill
  \end{frame}
}
\setbeamercolor{block title}{bg=myNewColorA, fg=black} % Background and foreground colors for the block title
\setbeamercolor{block body}{bg=myNewColorC, fg=black} % Background and foreground colors for the block body
%------------------------------------------------------------
\begin{document}
%The next statement creates the title page.
\frame{\titlepage}
\begin{frame}
\frametitle{Contents}
\tableofcontents
\end{frame}


%------------------------------------------------------------
\begin{frame}{Critical Systems and Applications of RTOS}
    \begin{itemize}
        \item Discussion of RTOS applications in critical systems across various industries
        \begin{itemize}
            \item \textbf{Aerospace and Aviation} (e.g., General Electric, Hydra)
            \item \textbf{Automotive} (e.g., Continental, NXP, Avnet)
            \item \textbf{Healthcare} (e.g., Plexus, Baxter)
            \item \textbf{Power Generation} (e.g., Baker Hughes, CERN)
            \item \textbf{Telecommunications} (e.g., Cinvestav, Qualcomm)
            \item \textbf{Defense and Military} (e.g., Hydra, A2E)
        \end{itemize}
    \end{itemize}
\end{frame}
%------------------------------------------------------------

%------------------------------------------------------------
\section{Aerospace and Aviation}
\begin{frame}{General Electric}
    \begin{itemize}
        \item Flight control systems, engine management, and in-flight entertainment systems.
        \item Santiago de Queretaro \href{https://www.geaerospace.com/commercial}{General Electric Aeroespace}
        \item Europe Aeroespace \href{https://www.dlr.de/en/careers/your-entry/your-specialist-field/research-funding-knowledge-management}{DLR}
        \item Hydra technologies \href{https://www.hydra-technologies.com/}{Hydra}
        \item A2E technologies (Guadalajara) \href{https://www.a2etechnologies.com/}{A2E technologies}
        \item DO-178C Software Considerations in Airborne Systems and Equipment Certification.
        \item Examples include managing real-time data from multiple sensors and executing complex algorithms for autopilot systems.
    \end{itemize}
\end{frame}

\begin{frame}{Hydra}

\end{frame}
\begin{frame}{Hydra}

\end{frame}
%------------------------------------------------------------
\section{Automotive Industry}
\begin{frame}{Automotive Industry}
    \begin{itemize}
        \item AUTOSAR (Automotive Open System ARchitecture), provides a standardized architecture for automotive software, facilitating the integration of RTOS into diverse automotive applications.
        \item Engine controls, safety systems, and infotainment systems.
        \item Advanced driver-assistance systems (ADAS) and autonomous driving technologies.
        \item \href{https://www.autosar.org/standards/classic-platform}{AutoSAR stack}
    \end{itemize}
\end{frame}
%------------------------------------------------------------
\section{Healthcare}
\begin{frame}{Healthcare}
    \begin{itemize}
        \item Healthcare systems rely on RTOS for the management of critical medical devices such as ventilators, infusion pumps, and diagnostic systems.
        \item These systems require absolute reliability and precision, where timing is often critical to patient care outcomes.
        \item RTOS ensure that the medical devices operate under strict timing constraints, providing consistent and reliable performance.
    \end{itemize}
\end{frame}
%------------------------------------------------------------

\section{Power Generation}
\begin{frame}{Power Generation}
    \begin{itemize}
        \item \href{https://www.bakerhughes.com/bently-nevada/monitoring-systems/machinery-protection/orbit-60-series}{Orbit60}
        \item RTOS are integral to managing operations in power generation and distribution, ensuring stability and efficiency in the power grid.
        \item They handle real-time monitoring and control of power systems, adjusting loads and generation in response to real-time demand changes.
        \item This real-time capability is vital for integrating renewable energy sources effectively, where power supply can be intermittent and variable.
    \end{itemize}
\end{frame}
%------------------------------------------------------------
\section{Telecommunications}
\begin{frame}{Telecommunications}
    \begin{itemize}
        \item In telecommunications, RTOS manage critical real-time operations such as signal processing, data routing, and session management.
        \item Their ability to handle high-speed data with low latency supports the reliability and quality of voice, video, and data transmission across networks.
        \item RTOS also enhance the scalability and adaptability of network infrastructures to meet changing data traffic demands.
    \end{itemize}
\end{frame}
%------------------------------------------------------------
\section{Defense and Military}
\begin{frame}{Defense and Military}
    \begin{itemize}
        \item Compliance with ITAR (International Traffic in Arms Regulations) is crucial, ensuring that defense technologies including those powered by RTOS adhere to strict export control laws, maintaining national security.
        \item The defense sector utilizes RTOS for systems requiring robust, real-time responses such as weaponry, surveillance, and communication systems.
        \item \href{https://www.geaerospace.com/military-defense/engines/xa100}{General Electric Aeroespace}
    \end{itemize}
\end{frame}
%------------------------------------------------------------

\section*{Acknowledgement}  
\begin{frame}
\textcolor{myNewColorA}{\Huge{\centerline{Thank you!}}}
\end{frame}


\end{document}